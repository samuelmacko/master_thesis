\chapter*{Introduction}
\addcontentsline{toc}{chapter}{Introduction}

This thesis was developed in collaboration with \emph{Red Hat}\footnote{See \url{https://www.redhat.com/en}} as a research project.
Partners from \emph{Red Hat} provided expert guidance throughout the whole process of the thesis development and infrastructure for the dataset gathering and model training.
% todo
(Mal by v uvode byt tento odstavec? Ak ano tak kde?)

Open source software (OSS) is a type of computer software in which source code is released under a license in which the copyright holder grants users the rights to use, study, change, and distribute the software to anyone and for any purpose\footnote{See \url{https://books.google.cz/books?id=04jG7TTLujoC&pg=PA4&redir_esc=y\#v=onepage&q&f=false}}.
Open source projects have an increasing relevance in modern software development.
For example, many critical software systems are currently available under open source licenses, including operating systems, compilers, databases, and web servers.
Similarly, it is common nowadays to depend on open source libraries and frameworks when building and evolving proprietary software \cite{p:7}.

\emph{GitHub}\footnote{See \url{https://github.com/}} is the world’s largest collection of open source software, with more than 56 million users and 100 million projects.
It offers the distributed version control and source code management functionality of \emph{Git}\footnote{See \url{https://git-scm.com/}} and some more features.

\subsection*{Aim}

As more and more technologies rely on open source software, it is important to put a lot of attention into choosing such dependencies.
If an open source repository ceases to be actively developed and becomes obsolete, it can cause serious issues for all of the other projects that depend on this repository.
It would be helpful to be able to determine if a \emph{GitHub} hosted repository has a high chance of being actively developed in the future, or if it will likely die off.
The aim of this thesis is precisely that, to develop a \emph{machine learning}\footnote{See \url{https://en.wikipedia.org/wiki/Machine_learning}} model to estimate a probability of survival of \emph{GitHub} repositories.

\subsection*{Key contributions}

I developed applications \ref{sec:applications} for gathering relevant data that can be used to construct an input dataset for \emph{machine learning} models \ref{sec:training_evaluation}.
I also proposed and tested a set of features \ref{chap-2:features} for determining project's survivability.

\subsection*{Contents}

Chapter \ref{chap-1:review} goes over some research papers aiming to solve a similar problem as this thesis.
In the chapter \ref{chap-2:features}, I go over proposed features and give an explanation of what they measure and why I decided to include them.
Chapter \ref{chap-3:implementation} describes the process of gathering and preparing data, as well as methods used for training and evaluating machine learning models used for making predictions.
Then, the chapter \ref{chap-4:results} provides a summary and analysis of the achieved results.
The final chapter, chapter \ref{chap-5:summary}, provides a summary of the thesis.
